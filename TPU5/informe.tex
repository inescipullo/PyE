\documentclass[11pt]{article}
\usepackage{graphicx}
\usepackage{fancyhdr}
% \usepackage{wrapfig}
\usepackage{hyperref}
% \usepackage{tabularx}
\usepackage{setspace}

% math package
\usepackage{amsmath}

\usepackage{yhmath}


\usepackage[spanish]{babel}

\newsavebox\CBox
\def\textBF#1{\sbox\CBox{#1}\resizebox{\wd\CBox}{\ht\CBox}{\textbf{#1}}}

\newenvironment{myenv}[1]
  {\begin{spacing}{#1}}
  {\end{spacing}}

\addtolength{\textwidth}{0.2cm}
\setlength{\parskip}{13pt}
\setlength{\parindent}{0.0cm}
\linespread{1.25}

\pagestyle{fancy}
\fancyhf{}
\rhead{TP - Cipullo, Sullivan}
\lhead{Probalidad y Estad\'istica}
\rfoot{\vspace{1cm} \thepage}

\renewcommand*\contentsname{\LARGE Índice}

\begin{document}

\begin{titlepage}
    \begin{center}
        \vfill
        \vfill
            \vspace{0.7cm}
            \noindent\textbf{\Huge Trabajo Pr\'actico Unidad 5}\par
            \noindent\textbf{\Huge Probabilidad y Estad\'istica}\par
            \vspace{.5cm}
        \vfill
        \noindent \textbf{\huge Alumnas:}\par
        \vspace{.5cm}
        \noindent \textbf{\Large Cipullo, In\'es}\par
        \noindent \textbf{\Large Sullivan, Katherine}\par
 
        \vfill
        \large Universidad Nacional de Rosario \par
        \noindent\large 2021
    \end{center}
\end{titlepage}
\par


\textbf{Ejercicio 1}

Contamos con cierto tipo de marcapasos cuya vida \'util sigue una distribuci\'on exponencial con media 16 a\~{n}os.
Por contar con una distribuci\'on exponencial, la media, notada con $E(X)$, se expresa como sigue: 
$$E(X) = \int_{0}^{+ \infty} x \alpha e^{-\alpha x} dx = \frac{1}{\alpha} = 16 \implies \alpha = \frac{1}{16}$$

Debemos calcular la probabilidad de que un marcapasos dure menos de 20 a\~{n}os. Esto equivale a calcular la funci\'on de distribuci\'on acumulada en $x=20$. Dicha funci\'on para una distribuci\'on queda definida como sigue: 
$$F(x) = \int_{0}^{x} \alpha e^{-\alpha t} dt = 1 - e^{-\alpha x}$$
Por lo tanto, la probabilidad buscada es: $$F(20) = 1 - e^{-\frac{1}{16} 20} = 1 - e^{\frac{-20}{16}} = 0.7135$$

Por otro lado, se solicita obtener la probabilidad de que un marcapasos dure menos de 25 a\~{n}os dado que lleva funcionando 5 a\~{n}os.

Por la propiedad de falta de memoria, caracter\'istica de la distribuci\'on exponencial, sabemos que: $$P(X > 20 + 5 / X > 5) =  P(X > 20),$$ 
es decir, la probabilidad de que este marcapasos dure m\'as de 25 a\~{n}os dado que ya dur\'o 5, es igual a la probabilidad de que otro marcapasos cualquiera dure m\'as de 20 a\~{n}os.

Ahora bien, nosotras debemos calcular la probabilidad de que este marcapasos dure menos de 25 a\~{n}os. Esto se reduce a $1 - P(X>20) = F(20) = 0.7135.$ 

\textbf{Ejercicio 2}

Se cuenta un láser semiconductor a potencia constante cuya duraci\'on $X$ tiene una distribuci\'on normal con media 7000 horas y desviaci\'on t\'ipica de 600 horas.

\textbf{a)}

Se solicita calcular la probabilidad de que el l\'aser dure menos de 5000 horas, lo cual es equivalente a evaluar la funci\'on de distribuci\'on acumulada en $x=5000$. Para obtener este valor, primero estandarizamos la variable $x$, siendo ahora nuestra variable de inter\'es $z = \frac{x - \mu}{\sigma}$, y luego buscamos el valor correspondiente a $F(\frac{5000-7000}{600}) = F(3.\wideparen{33})$ en la tabla de probabilidades de distribuci\'on normal. 
Llegando al resultado: $$P(X<5000) = 0.0004$$

\textbf{b)}

Debemos encontrar la duraci\'on excedida por el 95\% de los l\'aseres, es decir, debemos encontrar un valor $t$ (en horas) tal que $P(X>t) = 0.95$. Sabemos que $P(X>t) = 1 - F(z)$, donde $z = \frac{t - \mu}{\sigma}$ (variable estandarizada). Entonces si $P(X>t) = 0.95$ eso implica que $F(z) = 0.05$. 
Buscamos el valor $z$ dentro de la tabla de probabilidades de la distribuci\'on normal tal que $F(z) = 0.05$, el cual es $z = -1.64$. Luego desestandarizamos la variable $z$, obteniendo que: 
$$z = \frac{t - 7000}{600} = -1.64 \implies t = 600z + 7000 = 600 \cdot (-1.64) + 7000 = 6016$$

\textbf{c)}

Dados tres l\'aseres que fallan de manera independiente, buscamos la probabilidad de que los tres sigan funcionando luego de 7000 horas. Definimos la variable aleatoria discreta $Y$: ``n\'umero de l\'aseres que duran m\'as de 7000 horas de un total de 3 l\'aseres".
Sabiendo que la duraci\'on de cada l\'aser es independiente, queremos calcular $P(Y=3)$.

Para realizar esto, primero calculamos $P(X>7000) = 1 - F(\frac{7000-7000}{600}) = 1 - F(0) = 1 - 0.5 = 0.5$.

Como $Y$ cuenta con una distribuci\'on binomial con $n=3$ y la probabilidad asociada al suceso de inter\'es $p=P(X>7000)=0.5$, obtenemos que:
$$P(Y=3) = \binom{3}{3} \cdot 0.5^3 \cdot (1 - 0.5)^0 = 0.125$$ 

\textbf{Ejercicio 3}

Tenemos una variable aleatoria continua $T_c$ con distribuci\'on uniforme en el intervalo $(15,21)$, por lo tanto su funci\'on de densidad es $f_c(x) = \frac{1}{6}$.

Definimos una nueva variable aleatoria continua $T_f = H(T_c) = \frac{9}{5} \cdot T_c + 32$. Analizamos los valores posibles para $T_f$ a partir de los valores de $T_c$: 
$$15 < T_c < 21 \iff 15 \cdot \frac{9}{5} + 32 < T_c \cdot \frac{9}{5} + 32 < 21 \cdot \frac{9}{5} + 32 \iff 59 < T_f < 69.8$$ 
Al ser una funci\'on lineal creciente, podemos afirmar que $H$ es una funci\'on mon\'otona y derivable en los reales. 

Por lo tanto, la funci\'on de densidad de $T_f$ es: $$f_f(y) = f_c(H^{-1}(y)) \cdot \left|\frac{\partial H^{-1}(y)}{\partial y}\right|$$
Tenemos que $H^{-1}(y) = (y - 32) \cdot \frac{5}{9}$, por lo que $$59 < y < 69.8 \implies 15 < H^{-1}(y) < 21 \implies f_c(H^{-1}(y)) = \frac{1}{6}$$
Adem\'as, $(H^{-1})'(y) = \frac{5}{9}$. Entonces, resulta: 
$$f_f(y) = \frac{1}{6} \cdot \frac{5}{9} = \frac{5}{54}$$



\end{document} 