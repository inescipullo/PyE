\documentclass[11pt]{article}
\usepackage{graphicx}
\usepackage{fancyhdr}
% \usepackage{wrapfig}
\usepackage{hyperref}
% \usepackage{tabularx}
\usepackage{setspace}

% para tablas
\usepackage{multirow}
\usepackage{array}

% math package
\usepackage{amsmath}

\usepackage[spanish]{babel}

\newsavebox\CBox
\def\textBF#1{\sbox\CBox{#1}\resizebox{\wd\CBox}{\ht\CBox}{\textbf{#1}}}

\newenvironment{myenv}[1]
  {\begin{spacing}{#1}}
  {\end{spacing}}

\addtolength{\textwidth}{0.2cm}
\setlength{\parskip}{13pt}
\setlength{\parindent}{0.0cm}
\linespread{1.25}

\pagestyle{fancy}
\fancyhf{}
\rhead{TP - Cipullo, Sullivan}
\lhead{Probalidad y Estad\'istica}
\rfoot{\vspace{1cm} \thepage}

\renewcommand*\contentsname{\LARGE Índice}

\begin{document}

\begin{titlepage}
    \begin{center}
        \vfill
        \vfill
            \vspace{0.7cm}
            \noindent\textbf{\Huge Trabajo Pr\'actico Unidad 4}\par
            \noindent\textbf{\Huge Probabilidad y Estad\'istica}\par
            \vspace{.5cm}
        \vfill
        \noindent \textbf{\huge Alumnas:}\par
        \vspace{.5cm}
        \noindent \textbf{\Large Cipullo, In\'es}\par
        \noindent \textbf{\Large Sullivan, Katherine}\par
 
        \vfill
        \large Universidad Nacional de Rosario \par
        \noindent\large 2021
    \end{center}
\end{titlepage}
\par


\textbf{Ejercicio 1}

\textbf{A.} \par
Tenemos una variable aleatoria $X$, tal que $X$:”número de interrupciones diarias” en una compañía. 
Debemos calcular los valores $P(X=4)$ y $P(X=5)$. Sabemos que:  

\begin{itemize}
    \item $F(4) = 0.97$
    \item $P(X=0) = 0.32$ 
    \item $P(X=1) = 0.35$
    \item $P(X=2) = 0.18$
    \item $P(X=3) = 0.08$
    \item $P(X=6) = 0.01$
\end{itemize}

Primero, tenemos que: 
\begin{align*}
    F(4)& = P(X=0) + P(X=1) + P(X=2) + P(X=3) + P(X=4) =\\
        & = 0.32 + 0.35 + 0.18 + 0.08 + P(X=4) = 0.93 + P(X=4) =\\
        & = 0.97 \implies P(X=4) = 0.04
\end{align*}

Luego, como la variable aleatoria $X$ puede tomar unicamente valores enteros entre el 0 y el 6 inclusive, es decir, $P(X=x) = 0\ \forall\ x > 6$, y sabemos que la suma total de su distribución de probabilidades suma 1, tenemos que:
\begin{align*}
    &P(X=0) + P(X=1) + P(X=2) + P(X=3) + P(X=4) + P(X=5) + P(X=6) = \\
    &= 0.32 + 0.35 + 0.18 + 0.08 + 0.04 + P(X=5) + 0.01 = 0.98 + P(X=5) = 1 \implies \\
    &\implies P(X=5) = 0.02 
\end{align*}


\textbf{B.} \par
La probabilidad de que en un día dado haya a lo sumo 4 interrupciones refiere a la probabilidad de que la variable aleatoria $X$ tome un valor igual a 4 o menor, por lo tanto esto es $F(4) = 0.97$. \par
La probabilidad de que en un día dado haya por lo menos 5 interrupciones refiere a la probabiilidad de que la variable aleatoria $X$ tome un valor igual a 5 o mayor, por lo tanto esto es $P(X=5) + P(X=6) = 0.02 + 0.01 = 0.03$.


\textbf{C.} \par
Esperanza matemática de $X$:
\begin{align*}
    E(X)& = \sum_{i=1}^{7} x_i \cdot p(x_i) = \sum_{i=1}^{7} (i-1) \cdot p(i-1) = \\
        & = 0.35 + 2 \cdot 0.18 + 3 \cdot 0.08 + 4 \cdot 0.04 + 5 \cdot 0.02 + 6 \cdot 0.01 = \\
        & = 0.35 + 0.36 + 0.24 + 0.16 + 0.1 + 0.06 = 1.27
\end{align*}

Desviación estándar de $X$:
\begin{align*}
    \sigma_x = \sqrt{V(X)} = \sqrt{E(X^2) - E(X)^2},
\end{align*}
donde
\begin{align*}
    E(X)^2 & = 1.27^2 = 1.6129 \\
    E(X^2) & = \sum_{i=1}^{7} x_i^2 \cdot p(x_i) = \sum_{i=1}^{7} (i-1)^2 \cdot p(i-1) = \\
           & = 0.35 + 4 \cdot 0.18 + 9 \cdot 0.08 + 16 \cdot 0.04 + 25 \cdot 0.02 + 36 \cdot 0.01 = \\
           & = 0.35 + 0.72 + 0.72 + 0.64 + 0.5 + 0.36 = 3.29
\end{align*}
Volviendo a la fórmula, tenemos que:
\begin{align*}
    \sigma_x = \sqrt{3.29 - 1.6129} = \sqrt{1.6771} = 1.295
\end{align*}

\textbf{D.} \par

La esperanza matemática representa el valor medio que toma la variable aleatoria, en este caso 1.27.
La desviación estándar, por otro lado, indica que tan dispersos están los datos con respecto a la media, al ser en este caso relativamente baja indica que no hay una disperción de los datos tan significativa.


\textbf{Ejercicio 2}



\textbf{Ejercicio 3}

Sabemos que el 30\% de los aspirantes a un trabajo tienen entrenamiento avanzado en programación. Los aspirantes son entrevistados al azar y en forma sucesiva. \par
Consideramos que la cantidad total de aspirantes es muy grande, potencialmente infinita. Por lo tanto, al entrevistar una persona, independientemenete de si su entrenamiento es avanzado o no, sabremos que la proporción de aspirantes con entrenamineto avanzado en el grupo restante de aspirantes se mantiene (siempre será 30\%). Es decir, nos enfrentamos a una situación de extracción con reposición. \par

\textbf{A.} \par
Para calcular la probabilidad de haber encontrado el primer aspirante con entrenamiento avanzado en programación para la quinta entrevista utilizamos la distribución geométrica: \par
$P(X=k) = (1-p)^{k-1} \cdot p$, donde $k = 5$ y $p = 0.3$ \par
$P(X=5) = 0.7^4 \cdot 0.3 = 0.2401 \cdot 0.3 = 0.07203 $


\textbf{B.} \par
Para calcular la probabilidad de haber encontrado el quinto aspirante con entrenamiento avanzado en programación para la décima entrevista utilizamos la distribución de Pascal: \par
$P(Y=k) = \binom{k-1}{r-1} \cdot p^r \cdot (1-p)^{k-r}$, donde $k = 10$, $r = 5$ y $p = 0.3$ \par
$P(Y=10) = \binom{9}{5} \cdot 0.3^5 \cdot 0.7^5 = \frac{9!}{4! \cdot 5!} \cdot 0.3^5 \cdot 0.7^5 = 126 \cdot 0.00243 \cdot 0.16807 = 0.0515$


\end{document} 