\documentclass[11pt]{article}
\usepackage{graphicx}
\usepackage{fancyhdr}
% \usepackage{wrapfig}
\usepackage{hyperref}
\usepackage{tabularx}
\usepackage{setspace}

% math package
\usepackage{amsmath}
\usepackage{amssymb}

%\usepackage{yhmath}


\usepackage[spanish]{babel}
\decimalpoint

\newsavebox\CBox
\def\textBF#1{\sbox\CBox{#1}\resizebox{\wd\CBox}{\ht\CBox}{\textbf{#1}}}

\newenvironment{myenv}[1]
  {\begin{spacing}{#1}}
  {\end{spacing}}

\addtolength{\textwidth}{0.2cm}
\setlength{\parskip}{13pt}
\setlength{\parindent}{0.0cm}
\linespread{1.25}

\pagestyle{fancy}
\fancyhf{}
\rhead{TP - Cipullo, Sullivan}
\lhead{Probalidad y Estad\'istica}
\rfoot{\vspace{1cm} \thepage}

\renewcommand*\contentsname{\LARGE Índice}

\begin{document}

\begin{titlepage}
    \begin{center}
        \vfill
        \vfill
            \vspace{0.7cm}
            \noindent\textbf{\Huge Trabajo Pr\'actico Unidad 6}\par
            \noindent\textbf{\Huge Probabilidad y Estad\'istica}\par
            \vspace{.5cm}
        \vfill
        \noindent \textbf{\huge Alumnas:}\par
        \vspace{.5cm}
        \noindent \textbf{\Large Cipullo, In\'es}\par
        \noindent \textbf{\Large Sullivan, Katherine}\par
 
        \vfill
        \large Universidad Nacional de Rosario \par
        \noindent\large 2021
    \end{center}
\end{titlepage}
\par


\textbf{Ejercicio 1}

Sean las siguientes variables aleatorias:
\begin{itemize}
    \item $G_i$: ``resultado de la apuesta $i$''
    \item $G_t$: ``suma de los resultados de las primeras 50 apuestas''

\end{itemize}

Tenemos que:
\begin{itemize}
    \item $P(G_i = 0) = 0.9$
    \item $P(G_i = 1) = 0.1$
    \item $G_t = \sum_{i = 1}^{50} G_i$
\end{itemize}

Considerando que el resultado de una apuesta es independiente del resultado del resto de apuestas y que $E(G_i)$ y $V(G_i)$ son finitas 
para todo $i=1,...,50$ podemos aplicar el teorema central del l\'imite y concluir que la variable $G_t$ tiene una distribuci\'on 
aproximadamente normal de media $\mu$ y varainza $\sigma^2$, donde:

\begin{itemize}
    \item $\mu = E(G_t) = 50\cdot E(G_i) = 50\cdot 0.1 = 5$ 
    \item $\sigma^2 = V(G_t) = 50\cdot V(G_i) = 50\cdot 0.09 = 4.5$
\end{itemize}

Por lo tanto, 
$ G_t \sim N(5,4.5)$

Sea $S$ la cantidad de dinero con que la persona queda luego de las 50 apuestas. Queremos ver que $S\geq 0$. 

Recordando que se apuesta una cantidad de dinero $c > 0$ que se pierde en caso de que $G_i=0$ y que si $G_i=1$ se ganan $5c$, tenemos que:

\begin{align*}
    S &= 5\cdot c \cdot G_t - c\cdot (50 - G_t) \\
      &= 5\cdot c \cdot G_t - (c \cdot 50 + c\cdot G_t) \\
      &= 5\cdot c \cdot G_t + c\cdot G_t - c\cdot 50 \\
      &= 6\cdot c \cdot G_t - c\cdot 50
\end{align*}

Queremos que $S \geq 0 \implies 6c \ G_t - 50c \geq 0 \implies G_t \geq \frac{50}{6}$  

Entonces, en s\'intesis lo que buscamos es $P(G_t\geq 8.33)$. 

Para obtener este valor estandarizamos la variable:

$ Z = \frac{G_t - \mu}{\sigma} = \frac{G_t - 5}{\sqrt{4.5}} = \frac{G_t - 5}{2.12}$

Y por lo tanto,

$P(G_t\geq 8.33) = P(Z \geq 1.57) = 1 - P(Z < 1.57) = 1 - 0.9418 = 0.0582$


\textbf{Ejercicio 2}

Sean las variables aleatorias:
\begin{itemize}
    \item $A$: ``longitud de bloques de tipo A''
    \item $B$: ``longitud de bloques de tipo B''
    \item $X$: ``suma de la longitud de 20 bloques de tipo A''
    \item $Y$: ``suma de la longitud de 30 bloques de tipo B''
    \item $R$: ``longitud del recipiente''
\end{itemize}

Contamos con los siguientes datos:
\begin{itemize}
    \item $E(A) = 1.95$ y $\sqrt{V(A)} = 0.01 \implies V(A) = 0.0001$
    \item $E(B) = 0.83$ y $\sqrt{V(B)} = 0.02 \implies V(B) = 0.0004$
    \item $E(R) = 65$ y $\sqrt{V(R)} = 0.5 \implies V(R) = 0.7071$
\end{itemize}

Como $A$ y $B$ son variables aleatorias independientes con media y varianza finita, por el Teorema Central del L\'imite sabemos que las variables aleatorias $X$ e $Y$ tienen una distribuci\'on aproximadamente normal, donde $E(X) = 39$, $V(X) = 0.002$, $E(Y) = 24.9$ y $V(Y) = 0.012$.

Buscamos la probabilidad de que el ensamble formado por 20 bloques de tipo A y 30 bloques de tipo B entre en un recipiente. Podemos plantearlo como sigue:
$$P(X+Y \leq R) = P(X+Y-R \leq 0)$$

Como $R$ tiene distribuci\'on normal y $X$ e $Y$ tienen distribuci\'on aproximadamente normal, por la propiedad reproductica de la distribuci\'in normal:
$$X + Y + (-1) \cdot R \sim N(E(X)+E(Y)-E(R), \sqrt{V(X)+V(Y)-V(R)})$$

Definimos entonces una nueva variable aleatoria: $S = X+Y-R$, donde
$$S \sim N(-1.1, -0.6931)$$

En s\'intesis, queremos ver que $P(S \leq 0)$. Para obtener ese valor, debemos estandarizar la variable:

$Z = \frac{S - \mu}{\sigma^2} = \frac{S + 1.1}{-0.6931}$

Buscamos en la tabla de distribuci\'on normal est\'andar
$$P(S \leq 0) = P(Z \leq -1.587) = 0.0559$$


\textbf{Ejercicio 3}

Sean las variables aleatorias:
\begin{itemize}
    \item $X$: ``n\'umero de defectos tipo D1 que presenta una pieza''
    \item $Y$: ``n\'umero de defectos tipo D2 que presenta una pieza''
\end{itemize}
Contamos con los siguientes datos:
\begin{enumerate}
    \item $E(X) = 0.3$ y $V(X) = 0.21$
    \item $E(Y) = 0.8$ y $V(Y) = 0.56$
    \item 20\% de piezas tienen 2 defectos tipo D2
    \item 15\% de las piezas tienen 1 defecto tipo D1 y ninguno tipo D2
    \item 50\% de las piezas que no tienen defectos tipo D1, tienen 1 defecto tipo D2
\end{enumerate}

\textbf{a)}

\begin{center}
    \begin{tabularx} {0.8\textwidth}{ 
        | >{\raggedright\arraybackslash}X 
        | >{\raggedleft\arraybackslash}X 
        | >{\raggedleft\arraybackslash}X 
        | >{\raggedleft\arraybackslash}X 
        | >{\raggedleft\arraybackslash}X | }
        \hline
        \textbf{X \textbackslash Y} & \textbf{0} & \textbf{1} & \textbf{2} & \textbf{$p_X(x)$} \\
        \hline
        \textbf{0}                  & 0.25       & 0.35       & 0.1        & 0.7 \\
        \hline
        \textbf{1}                  & 0.15       & 0.05       & 0.1        & 0.3 \\
        \hline
        \textbf{$p_Y(y)$}           & 0.4        & 0.4        & 0.2        & 1 \\
        \hline
   \end{tabularx}
\end{center}

C\'alculos realizados: 
\begin{itemize}
    \item $p(1,0) = 0.15$ (por lo especificado en el dato 4)
    \item $p_Y(2) = 0.2$ (por lo especificado en el dato 3)
    \item $E(X) = 0 \cdot p_X(0) + 1 \cdot p_X(1) = p_X(1) \implies p_X(1) = 0.3$
    \item $p_X(0) + p_X(1) = p_X(0) + 0.3 = 1 \implies p_X(0) = 0.7$
    \item $p(0,1) = p_X(0) \cdot 0.5 = 0.7 \cdot 0.5 = 0.35$ (por lo especificado en el dato 5)
    \item $E(Y) = 0 \cdot p_Y(0) + 1 \cdot p_Y(1) + 2 \cdot p_Y(2) = p_Y(1) + 2 \cdot 0.2 \implies p_Y(1) = 0.4$
    \item $p_Y(0) + p_Y(1) + p_Y(2) = p_Y(0) + 0.4 + 0.2 = 1 \implies p_Y(0) = 0.4$
    \item $p_Y(0) = p(0,0) + p(1,0) \implies 0.4 = p(0,0) + 0.15 \implies p(0,0) = 0.25$
    \item $p_X(0) = p(0,0) + p(0,1) + p(0,2) \implies 0.7 = 0.25 + 0.35 + p(0,2) \implies p(0,2) = 0.1$
    \item $p_Y(1) = p(0,1) + p(1,1) \implies 0.4 = 0.35 + p(1,1) \implies p(1,1) = 0.5$
    \item $p_Y(2) = p(0,2) + p(1,2) \implies 0.2 = 0.1 + p(1,2) \implies p(1,2) = 0.1$
\end{itemize}


\textbf{b)}

Las variables aleatorias $X$ e $Y$ son independientes si $$p(x,y) = p_X(x) \cdot p_Y(y) \ \forall\ (x,y) \in R_{(X \times Y)}$$
Notar que en este caso $R_{(X \times Y)} = \{(0,0),(0,1),(0,2),(1,0),(1,1),(1,2)\}$.

Pero tenemos que $p(0,0) = 0.25$ y $p_X(0) \cdot p_Y(0) = 0.7 \cdot 0.4 = 0.28$, por lo tanto $X$ e $Y$ no son independientes. 

Calculamos entonces el Coeficiente de Correlaci\'on:

\begin{align*}
    \rho_{xy} &= \frac{Cov(X,Y)}{\sqrt{V(X) \cdot V(Y)}} = \frac{E(X \cdot Y) - E(X) \cdot E(Y)}{\sqrt{V(X) \cdot V(Y)}} \\
              &\stackrel{(*)}{=} \frac{0.25 - 0.3 \cdot 0.8}{\sqrt{0.21 \cdot 0.56}} = 0.0292
\end{align*}
donde $(*)$ hace referencia al siguiente c\'alculo:
\begin{align*}
    E(X \cdot Y) &= \sum_{(x,y) \in R_{(X \times Y)}} x \cdot y \cdot p(x,y) \\
                 &= p(1,1) + 2 \cdot p(1,2) = 0.05 + 2 \cdot 0.1 = 0.25 
\end{align*}


\textbf{c)}

La f\'ormula de probabilidad condicional es como sigue:
$$P(Y=y / X=x) = \frac{p(x,y)}{p_X(x)}$$

Queda entonces que
$$P(Y=2 / X=0) = \frac{p(0,2)}{p_X(0)} = \frac{0.1}{0.7} = 0.1428$$

Este resultado refleja que dada una pieza que no tiene defectos tipo D1, la probabilidad de que tenga 2 defectos tipo D2 es $0.1428$. 
Otra forma de verlo es pensar que el $14.28\%$ de las piezas que no tienen defectos D1, tienen 2 defectos D2.


\textbf{d)}

Buscamos la esperanza y la varianza del costo de reparaci\'on por pieza, siendo que un defecto tipo D1 tiene un costo de reparaci\'on de \$3 y uno tipo D2 tiene un costo de reparaci\'on de \$4.

Definimos una funci\'on sobre las variables aleatorias: $Z = H(X,Y)$, donde $H(x,y) = 3 \cdot x + 4 \cdot y$.

Luego, basta con calcular $E(Z)$ y $V(Z)$.

\begin{align*}
    E(Z) &= \sum_{(x,y) \in R_{(X \times Y)}} H(x,y) \cdot p(x,y) \\ 
         &= 4 \cdot p(0,1) + 8 \cdot p(0,2) + 3 \cdot p(1,0) + 7 \cdot p(1,1) + 11 \cdot p(1,2) \\
         &= 4 \cdot 0.35 + 8 \cdot 0.1 + 3 \cdot 0.15 + 7 \cdot 0.05 + 11 \cdot 0.1 \\
         &= 1.4 + 0.8 + 0.45 + 0.35 + 1.1 = 4.1
\end{align*}

\begin{align*}
    V(Z) &= \sum_{(x,y) \in R_{(X \times Y)}} [H(x,y) - E(Z)]^2 \cdot p(x,y) \\
         &= (-4.1)^2 \cdot p(0,0) + (-0.1)^2 \cdot p(0,1) + 3.9^2 \cdot p(0,2) + \\ &\ \ \ + (-1.1)^2 \cdot p(1,0) + 2.9^2 \cdot p(1,1) + 6.9^2 \cdot p(1,2) \\
         &= 16.81 \cdot 0.25 + 0.01 \cdot 0.35 + 15.21 \cdot 0.1 + 1.21 \cdot 0.15 + 8.41 \cdot 0.05 + 47.61 \cdot 0.1 \\
         &= 4.2025 + 0.0035 + 1.521 + 0.1815 + 0.4205 + 4.761 = 11.09
\end{align*}


\end{document} 