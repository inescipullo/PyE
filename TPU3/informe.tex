\documentclass[11pt]{article}
\usepackage{graphicx}
\usepackage{fancyhdr}
% \usepackage{wrapfig}
\usepackage{hyperref}
% \usepackage{tabularx}
\usepackage{setspace}

% para tablas
\usepackage{multirow}
\usepackage{array}

\usepackage[spanish]{babel}

\newsavebox\CBox
\def\textBF#1{\sbox\CBox{#1}\resizebox{\wd\CBox}{\ht\CBox}{\textbf{#1}}}

\newenvironment{myenv}[1]
  {\begin{spacing}{#1}}
  {\end{spacing}}

\addtolength{\textwidth}{0.2cm}
\setlength{\parskip}{13pt}
\setlength{\parindent}{0.0cm}
\linespread{1.25}

\pagestyle{fancy}
\fancyhf{}
\rhead{TP - Cipullo, Sullivan}
\lhead{Probalidad y Estad\'istica}
\rfoot{\vspace{1cm} \thepage}

\renewcommand*\contentsname{\LARGE Índice}

\begin{document}

\begin{titlepage}
    \begin{center}
        \vfill
        \vfill
            \vspace{0.7cm}
            \noindent\textbf{\Huge Trabajo Pr\'actico Unidad 3}\par
            \noindent\textbf{\Huge Probabilidad y Estad\'istica}\par
            \vspace{.5cm}
        \vfill
        \noindent \textbf{\huge Alumnas:}\par
        \vspace{.5cm}
        \noindent \textbf{\Large Cipullo, In\'es}\par
        \noindent \textbf{\Large Sullivan, Katherine}\par
 
        \vfill
        \large Universidad Nacional de Rosario \par
        \noindent\large 2021
    \end{center}
\end{titlepage}
\par


\textbf{Ejercicio 1}

Considerando un canal de comunicaci\'on de un bit donde la probabilidad de enviar un 0 o un 1 es la misma, definimos los siguientes sucesos: 
\begin{itemize}
    \item Se emite un 0: $A$
    \item Se emite un 1: $B$
    \item Se recibe un 0: $C$
    \item Se recibe un 1: $D$
\end{itemize}

As\'i, definimos al suceso $Error$ (se comete error en la transmisión) como: que suceda $A$ y $D$ o suceda $B$ y $C$. Es decir, $$(A \cap D) \cup (B \cap D).$$

Buscamos entonces, calcular $P(Error) = P((A \cap D) \cup (B \cap D))$.

Por definci\'on de la probabilidad de la uni\'on, 
\[ P(Error) = P(A \cap D) + P(B \cap C) - P((A \cap D) \cap (B \cap C)) \]

Dado que $A \cap B = \emptyset$, sabemos que $P((A \cap D) \cap (B \cap C)) = 0$.

Entonces, por esto y por el teorema de la multipliaci\'on de probabilidades,
\[ P(Error) =  P(D/A) \cdot P(A) + P(C/B) \cdot P(B) - 0 = P(D/A) \cdot P(A) + P(C/B) \cdot P(B) \]

Por u\'ltimo, sabiendo que $P(A) = P(B) = 0.5$, podemos aplicar la propiedad distributiva:
\[ P(Error) = 0.5 \cdot (P(D/A) + P(C/B)) \]

Para poder asignarle un valor a la probabilidad de que ocurra un error, quedaría conocer los valores de $P(D/A)$, que se entiende como la probabilidad de que se cumpla $D$ dado que se sucedi\'o $A$ y $P(C/B)$, que se entinede como la probabilidad de que se cumpla $C$ dado que sucedi\'o $B$.


            
\textbf{Ejercicio 2}
 
a) Se solicita crear una tabla en donde cada celda represente la intersecci\'on de los sucesos de la primera prueba con los de la segunda prueba.

Esta se presenta a continuaci\'on. 

\vspace{0.5mm}

    \setlength{\extrarowheight}{2ex}
    \begin{center} 
    \centering 

        \begin{tabular}{|m{1.5cm}
                        |m{2cm}
                        |m{2cm}
                        |m{2cm}
                        |m{2cm}|}
        % {|c|c|c|c|c|}
        \hline
        \multicolumn{2}{|c|}{} &\multicolumn{3}{|c|}{Tipo de Error en la Segunda Prueba} \\
        \cline{3-5}
        \multicolumn{2}{|c|}{}  & Importante & Menor & Ninguno \\
        \hline
        \multirow{3.3}{*}{\rotatebox{90}{Tipo de error}} \multirow{3.3}{*}{\rotatebox{90}{en la Primera}} \multirow{3.3}{*}{\rotatebox{90}{Prueba}}
        & Importante & 0.18 & 0.3  & 0.12 \\ \cline{2-5}    
        & Menor      & 0.03 & 0.09 & 0.18 \\ \cline{2-5}
        & Ninguno    & 0    & 0.02 & 0.08 \\ \cline{2-5}
        \hline  
        \end{tabular}
    
    \end{center}


\vspace{0.5mm}


b) Se indica calcular la probabilidad de un error importante en la segunda prueba. 

Primero, otorg\'emosle nombre a los siguientes sucesos: 
\begin{itemize}
    \item Ocurre un error importante en la segunda prueba: $I_{2}$
    \item Ocurre un error importante en la primera prueba: $I_{1}$
    \item Ocurre un error menor en la primera prueba: $M_{1}$
    \item No ocurre ning\'un error en la primera prueba: $N_{1}$
\end{itemize}

Definimos $S$ como el espacio muestral asociado a la primera prueba.
Es decir, 
\[S=\{I_{1},M_{1},N_{1}\}\]

Veamos que los sucesos $I_{1}$, $M_{1}$ y $N_{1}$ representan una partición del espacio muestral $S$, puesto que: 
\begin{itemize}
    \item $I_{1} \cap  M_{1} = \emptyset$, $M_{1} \cap N_{1} = \emptyset$ y $N_{1} \cap I_{1} = \emptyset$, ya que solo puede ocurrir uno de los sucesos cuando se corre la primera prueba
    \item $I_{1} \cup M_{1} \cup N_{1} = S$, y
    \item $P(I_{1}) = 0.6 > 0$, $P(M_{1}) = 0.3 > 0$ y $P(N_{1}) = 0.1 > 0$. 
\end{itemize}

Luego como los sucesos representan una partici\'on del espacio $S$ podemos descomponer a $I_{2}$ como la uni\'on de las intersecciones de $I_{2}$ con estos sucesos (resultando una uni\'on de sucesos mutuamente excluyentes)
\[I_{2} = (I_{2} \cap I_{1}) \cup (I_{2} \cap M_{1}) \cup (I_{2} \cap N_{1}) \]

Y, entonces, la probabilidad de $I_{2}$ queda expresada como: 
\[P(I_{2}) = P(I_{2}\cap I_{1}) + P(I_{2}\cap M_{1}) + P(I_{2}\cap N_{1}) = 0.18 + 0.03 + 0 = 0.21 \]


c) Se requiere encontrar la probabilidad de error menor en la primera prueba sabiendo que el
error en la segunda prueba es importante.

Manteniendo los nombres otorgados en el inciso anterior lo que se debe encontrar es $P(M_{1}/I_{2})$

Aplicando la definici\'on de probabilidad condicional se tiene que:

\[P(M_{1}/I_{2}) = \frac{P(M_{1}\cap I_{2})}{P(I_{2})} = \frac{0.03}{0.21} = 0.1428\]

d) Se precisa analizar la independencia entre los resultados de la primera prueba y los resultados de la segunda prueba. 

Mantenemos los nombres otorgados en el inciso b) y llamamos $M_{2}$ a la ocurrencia de un error menor en la segunda prueba y $N_{2}$ a la no ocurrencia de un error en la segunda prueba. 

Lo que debemos analizar para ver si dos sucesos son independientes es si la probabilidad de su intersección es igual a la multiplicación de sus probabilidades individuales y eso es lo que haremos. 

\begin{itemize}
    \item \underline{$I_{2}$ con $I_{1}$}: Como $P(I_{2}\cap I_{1}) = 0.18$ y $P(I_{2})\cdot P(I_{1}) = 0.21\cdot 0.6 = 0.126$ los sucesos no son independientes. 
    \item \underline{$I_{2}$ con $M_{1}$}: Como $P(I_{2}\cap M_{1}) = 0.03$ y $P(I_{2})\cdot P(M_{1}) = 0.21\cdot 0.3 = 0.063$ los sucesos no son independientes. 
    \item \underline{$I_{2}$ con $N_{1}$}: Como $P(I_{2}\cap N_{1}) = 0$ y $P(I_{2})\cdot P(N_{1}) = 0.21\cdot 0.1 = 0.021$ los sucesos no son independientes. 
\end{itemize}

Antes de analizar la independencia de los resultados de la primer prueba con $M_{2}$ voy a querer analizar cu\'al es la probabilidad de $M_{2}$. Para eso, vamos a valernos de lo que expusimos en el inciso b). 

Como $I_{1}$, $M_{1}$ y $N_{1}$ representan una partici\'on del espacio muestral $S$ (definido en el inciso b), podemos descomponer a $M_{2}$ como la uni\'on de las intersecciones de $M_{2}$ con estos sucesos (resultando una uni\'on de sucesos mutuamente excluyentes)
\[M_{2} = (M_{2} \cap I_{1}) \cup (M_{2} \cap M_{1}) \cup (M_{2} \cap N_{1}) \]

Y, entonces, la probabilidad de $M_{2}$ queda expresada como: 
\[P(M_{2}) = P(M_{2}\cap I_{1}) + P(M_{2}\cap M_{1}) + P(M_{2}\cap N_{1}) = 0.3 + 0.09 + 0.02 = 0.41 \]

Entonces, continuamos con el an\'alisis de independencia,

\begin{itemize}
    \item \underline{$M_{2}$ con $I_{1}$}: Como $P(M_{2}\cap I_{1}) = 0.3$ y $P(M_{2})\cdot P(I_{1}) = 0.41\cdot 0.6 = 0.246$ los sucesos no son independientes. 
    \item \underline{$M_{2}$ con $M_{1}$}: Como $P(M_{2}\cap M_{1}) = 0.09$ y $P(M_{2})\cdot P(M_{1}) = 0.41\cdot 0.3 = 0.123$ los sucesos no son independientes. 
    \item \underline{$M_{2}$ con $N_{1}$}: Como $P(M_{2}\cap N_{1}) = 0.02$ y $P(M_{2})\cdot P(N_{1}) = 0.41\cdot 0.1 = 0.041$ los sucesos no son independientes. 
\end{itemize}

Y, como hicimos con $M_{2}$ antes de analizar la independencia de los resultados de la primer prueba con $N_{2}$ voy a querer analizar cu\'al es la probabilidad de $N_{2}$, valiendonos de lo que expusimos en el inciso b). 

Como $I_{1}$, $M_{1}$ y $N_{1}$ representan una partici\'on del espacio muestral $S$ (definido en el inciso b), podemos descomponer a $N_{2}$ como la uni\'on de las intersecciones de $N_{2}$ con estos sucesos (resultando una uni\'on de sucesos mutuamente excluyentes)
\[N_{2} = (N_{2} \cap I_{1}) \cup (N_{2} \cap M_{1}) \cup (N_{2} \cap N_{1}) \]

Y, entonces, la probabilidad de $N_{2}$ queda expresada como: 
\[P(N_{2}) = P(N_{2} \cap I_{1}) + P(N_{2} \cap M_{1}) + P(N_{2} \cap N_{1}) = 0.12 + 0.18 + 0.08 = 0.38 \]

Entonces, terminamos con el an\'alisis de independencia,

\begin{itemize}
    \item \underline{$N_{2}$ con $I_{1}$}: Como $P(N_{2}\cap I_{1}) = 0.12$ y $P(N_{2})\cdot P(I_{1}) = 0.38\cdot 0.6 = 0.228$ los sucesos no son independientes. 
    \item \underline{$N_{2}$ con $M_{1}$}: Como $P(N_{2}\cap M_{1}) = 0.18$ y $P(N_{2})\cdot P(M_{1}) = 0.38\cdot 0.3 = 0.114$ los sucesos no son independientes. 
    \item \underline{$N_{2}$ con $N_{1}$}: Como $P(N_{2}\cap N_{1}) = 0.08$ y $P(N_{2})\cdot P(N_{1}) = 0.38\cdot 0.1 = 0.038$ los sucesos no son independientes. 
\end{itemize}

\textbf{Ejercicio 3}

Tomando una persona de la poblaci\'on, denominamos a los siguientes sucesos: 
% \vspace{-0.5mm}
\begin{itemize}
    \item La persona padece la enfermedad: $E$
    \item La persona no padece la enfermedad: $\overline{E}$
    \item Le di\'o positiva la prueba: $D$
    \item Le di\'o negativa la prueba: $\overline{D}$
\end{itemize}

Se solicita calcular la probabilidad de que una persona a la que la prueba le ha dado positiva, est\'e sana. Es decir, se busca calcular $$P(\overline{E}/D).$$

Por definici\'on de probabilidad condicionada,

\[P(\overline{E}/D) = \frac{P(\overline{E}\cap D)}{P(D)}\]

Entonces intentaremos encontrar los valores de $P(D)$ y $P(\overline{E} \cap D)$.

Sabemos que: 
\begin{itemize}
    \item $P(E) = 0.12$
    \item $P(\overline{E}) = 0.88$
    \item $P(D/E) = 0.9$
    \item $P(D/\overline{E}) = 0.05$
\end{itemize}

Primero, vamos a calcular $P(D)$.

Sabiendo que la porbabilidad de la intersecci\'on entre dos sucesos es la probabilidad de que ambos ocurran, podemos expresar la probabilidad de $D$ como la probabilidad de la intersecci\'on entre $D$ y un suceso que va a pasar s\'i o s\'i, por ejemplo, $E\cup \overline{E}$. Es decir,
\[P(D) = P(D \cap (E \cup \overline{E}))\]

A su vez, sabemos que la probabilidad de que $E$ y $\overline{E}$ sucedan juntos es $0$, por lo tanto, 
\[P(E\cap \overline{E}) = 0\] y, consecuentemente, \[P(D\cap (E \cap \overline{E})) = 0\] 

Entonces, como esta probabilidad es $0$, la podemos aprovechar para la construcci\'on de $P(D)$: 
\[P(D) = P(D \cap (E \cup \overline{E})) + P(D\cap (E \cap \overline{E}))\]

Por propiedad de conjuntos, dados $A$, $B$ y $C$ conjuntos, 
\[A \cap (B\cup C) = (A\cap B) \cup (A\cap C)\]

Por lo tanto, podemos expandir nuestra expresi\'on de $P(D)$ haciendo uso de esta propiedad y del hecho que la intersecci\'on es conmutativa, sumado a que $D\cap D = D$: 
\[P(D) = P((D \cap E) \cup (D \cap \overline{E})) + P((D\cap E) \cap (D \cap \overline{E}))\]

Por propiedad, la probabilidad de la uni\'on de dos sucesos es la suma de sus probabilidades individuales menos la probabilidad de su intersecci\'on. Es decir dados 2 conjuntos $A$ y $B$, 
\[P(A\cup B)= P(A)+P(B)-P(A \cap B)\]
pero entonces tambi\'en resulta que 
\begin{equation}
P(A\cup B) + P(A\cap B) = P(A) + P(B)
\end{equation}

Entonces, reemplazando $A$ por $D\cap E$ y $B$ por $D\cap \overline{E}$ en $(1)$ la expresi\'on de $P(D)$ puede resultar como sigue
\[P(D) = P(D\cap E) + P(D\cap \overline{E})\]

Luego, aplicando el teorema de la multiplicaci\'on de
probabilidades tenemos que: 
\[P(D) = P(D/E)\cdot P(E) + P(D/\overline{E}) \cdot P(\overline{E})\]

Dado que ya cuento con los valores del lado izquierdo de la igualdad, concluyo que: 
\[P(D) = 0.108 + 0.044 = 0.152\]

Ahora, resta encontrar $P(\overline{E}\cap D)$. 

Por teorema de multiplicaci\'on de las probabilidades, 
\[P(\overline{E}\cap D) = P(D/\overline{E})\cdot P(\overline{E})\]

Como $P(D/\overline{E})$y $P(\overline{E})$ son valores que ya conozco, 

\[P(\overline{E}\cap D) = 0.05 \cdot 0.88 = 0.044\] 

Habiendo calculado los 2 datos que necesitabamos resulta que 

\[P(\overline{E}/D) = \frac{0.044}{0.152} = 0.2895\]


$$P(D/E) \cdot P(E) + P(D/\overline{E}) \cdot P(\overline{E}) = P(D \cap E) + P(D \cap \overline{E}) = $$
$$ = P((D \cap E) \cup (D \cap \overline{E})) + P((D \cap E) \cap (D \cap \overline{E})) = $$
$$ = P(D \cap (E \cup \overline{E})) + P(D \cap (E \cap \overline{E})) = P(D \cap (E \cup \overline{E})) + P(\emptyset) = $$
$$ = P(D \cap (E \cup \overline{E})) = P(D) $$



\end{document} 