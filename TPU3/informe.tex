\documentclass[11pt]{article}
\usepackage{graphicx}
\usepackage{fancyhdr}
\usepackage{wrapfig}
\usepackage{hyperref}
\usepackage{tabularx}
\usepackage{setspace}
\usepackage[spanish]{babel}

\newsavebox\CBox
\def\textBF#1{\sbox\CBox{#1}\resizebox{\wd\CBox}{\ht\CBox}{\textbf{#1}}}

\newenvironment{myenv}[1]
  {\begin{spacing}{#1}}
  {\end{spacing}}

\addtolength{\textwidth}{0.2cm}
\setlength{\parskip}{13pt}
\setlength{\parindent}{0.0cm}
\linespread{1.25}

\pagestyle{fancy}
\fancyhf{}
\rhead{TP - Cipullo, Sullivan}
\lhead{Probalidad y Estad\'istica}
\rfoot{\vspace{1cm} \thepage}

\renewcommand*\contentsname{\LARGE Índice}

\begin{document}

\begin{titlepage}
    \begin{center}
        \vfill
        \vfill
            \vspace{0.7cm}
            \noindent\textbf{\Huge Trabajo Pr\'actico Unidad 3}\par
            \noindent\textbf{\Huge Probabilidad y Estad\'istica}\par
            \vspace{.5cm}
        \vfill
        \noindent \textbf{\huge Alumnas:}\par
        \vspace{.5cm}
        \noindent \textbf{\Large Cipullo, In\'es}\par
        \noindent \textbf{\Large Sullivan, Katherine}\par
 
        \vfill
        \large Universidad Nacional de Rosario \par
        \noindent\large 2021
    \end{center}
\end{titlepage}
\par

\textbf{Ejercicio 2}
 
a) Se solicita crear una tabla en donde cda celda represente la intersecci\'on de los sucesos de la primera prueba con los de la segunda prueba.

Esta se presenta a continuaci\'on. 

b) Se indica calcular la probabilidad de un error importante en la segunda prueba. 

Primero, otorg\'emosle nombre a los siguientes sucesos: 
\begin{itemize}
    \item Ocurre un error importante en la segunda prueba: $I_{2}$
    \item Ocurre un error importante en la primera prueba: $I_{1}$
    \item Ocurre un error menor en la segunda prueba: $M_{1}$
    \item No ocurre ning\'un error en la primera prueba: $N_{1}$
\end{itemize}

Definimos $S$ como el espacio muestral asociado a la primera prueba.
Es decir, 
\[S=\{I_{1},M_{1},N_{1}\}\]

Veamos que los sucesos $I_{1}$, $M_{1}$ y $N_{1}$ representan una partición del espacio muestral $S$, puesto que: 
\begin{itemize}
    \item $I_{1} \cap  M_{1} = \emptyset$, $M_{1} \cap N_{1} = \emptyset$ y $N_{1} \cap I_{1} = \emptyset$, ya que solo puede ocurrir uno de los sucesos cuando se corre la primera prueba
    \item $I_{1} \cup M_{1} \cup N_{1} = S$, y
    \item $P(I_{1}) = 0.6 > 0$, $P(M_{1}) = 0.3 > 0$ y $P(N_{1}) = 0.1 > 0$. 
\end{itemize}

Luego como los sucesos representan una partici\'on del espacio $S$ podemos descomponer a $I_{2}$ como la uni\'on de las intersecciones de $I_{2}$ con estos sucesos (resultando una uni\'on de sucesos mutuamente excluyentes)
\[I_{2} = (I_{2} \cap I_{1}) \cup (I_{2} \cap M_{1}) \cup (I_{2} \cap N_{1}) \]

Y, entonces, la probabilidad de $I_{2}$ queda expresada como: 
\[P(I_{2}) = P(I_{2}\cap I_{1}) + P(I_{2}\cap M_{1}) + P(I_{2}\cap N_{1}) = 0.18 + 0.03 + 0 = 0.21 \]


c) Se requiere encontrar la probabilidad de error menor en la primera prueba sabiendo que el
error en la segunda prueba es importante.

Manteniendo los nombres otorgados en el inciso anterior lo que se debe encontrar es $P(M_{1}/I_{2})$

Aplicando la definici\'on de probabilidad condicional se tiene que:

\[P(M_{1}/I_{2}) = \frac{P(M_{1}\cap I_{2})}{P(I_{2})} = \frac{0.03}{0.21} = 0.0063\]

d) Se precisa analizar la independencia entre los resultados de la primera prueba y los resultados de la segunda prueba. 

Mantenemos los nombres otorgados en el inciso b) y llamamos $M_{2}$ a la ocurrencia de un error menor en la segunda prueba y $N_{2}$ a la no ocurrencia de un error en la segunda prueba. 

Lo que debemos analizar para ver si dos sucesos son independientes es si la probabilidad de su intersección es igual a la multiplicación de sus probabilidades individuales y eso es lo que haremos. 

\begin{itemize}
    \item \underline{$I_{2}$ con $I_{1}$}: Como $P(I_{2}\cap I_{1}) = 0.18$ y $P(I_{2})\cdot P(I_{1}) = 0.21\cdot 0.6 = 0.126$ los sucesos no son independientes. 
    \item \underline{$I_{2}$ con $M_{1}$}: Como $P(I_{2}\cap M_{1}) = 0.03$ y $P(I_{2})\cdot P(M_{1}) = 0.21\cdot 0.3 = 0.063$ los sucesos no son independientes. 
    \item \underline{$I_{2}$ con $N_{1}$}: Como $P(I_{2}\cap N_{1}) = 0$ y $P(I_{2})\cdot P(N_{1}) = 0.21\cdot 0.1 = 0.021$ los sucesos no son independientes. 
\end{itemize}

Antes de analizar la independencia de los resultados de la primer prueba con $M_{2}$ voy a querer analizar cu\'al es la probabilidad de $M_{2}$. Para eso, vamos a valernos de lo que expusimos en el inciso b). 

Como $I_{1}$, $M_{1}$ y $N_{1}$ representan una partici\'on del espacio muestral $S$ (definido en el inciso b), podemos descomponer a $M_{2}$ como la uni\'on de las intersecciones de $M_{2}$ con estos sucesos (resultando una uni\'on de sucesos mutuamente excluyentes)
\[M_{2} = (M_{2} \cap I_{1}) \cup (M_{2} \cap M_{1}) \cup (M_{2} \cap N_{1}) \]

Y, entonces, la probabilidad de $M_{2}$ queda expresada como: 
\[P(M_{2}) = P(M_{2}\cap I_{1}) + P(M_{2}\cap M_{1}) + P(M_{2}\cap N_{1}) = 0.3 + 0.09 + 0.02 = 0.41 \]

Entonces, continuamos con el an\'alisis de independencia,

\begin{itemize}
    \item \underline{$M_{2}$ con $I_{1}$}: Como $P(M_{2}\cap I_{1}) = 0.3$ y $P(M_{2})\cdot P(I_{1}) = 0.41\cdot 0.6 = 0.246$ los sucesos no son independientes. 
    \item \underline{$M_{2}$ con $M_{1}$}: Como $P(M_{2}\cap M_{1}) = 0.09$ y $P(M_{2})\cdot P(M_{1}) = 0.41\cdot 0.3 = 0.123$ los sucesos no son independientes. 
    \item \underline{$M_{2}$ con $N_{1}$}: Como $P(M_{2}\cap N_{1}) = 0.02$ y $P(M_{2})\cdot P(N_{1}) = 0.41\cdot 0.1 = 0.041$ los sucesos no son independientes. 
\end{itemize}

Y, como hicimos con $M_{2}$ antes de analizar la independencia de los resultados de la primer prueba con $N_{2}$ voy a querer analizar cu\'al es la probabilidad de $N_{2}$, valiendonos de lo que expusimos en el inciso b). 

Como $I_{1}$, $M_{1}$ y $N_{1}$ representan una partici\'on del espacio muestral $S$ (definido en el inciso b), podemos descomponer a $N_{2}$ como la uni\'on de las intersecciones de $N_{2}$ con estos sucesos (resultando una uni\'on de sucesos mutuamente excluyentes)
\[N_{2} = (N_{2} \cap I_{1}) \cup (N_{2} \cap M_{1}) \cup (N_{2} \cap N_{1}) \]

Y, entonces, la probabilidad de $M_{2}$ queda expresada como: 
\[P(N_{2}) = P(N_{2}\cap I_{1}) + P(N_{2}\cap M_{1}) + P(N_{2}\cap N_{1}) = 0.12 + 0.18 + 0.08 = 0.38 \]

Entonces, terminamos con el an\'alisis de independencia,

\begin{itemize}
    \item \underline{$N_{2}$ con $I_{1}$}: Como $P(N_{2}\cap I_{1}) = 0.12$ y $P(N_{2})\cdot P(I_{1}) = 0.38\cdot 0.6 = 0.228$ los sucesos no son independientes. 
    \item \underline{$N_{2}$ con $M_{1}$}: Como $P(N_{2}\cap M_{1}) = 0.18$ y $P(N_{2})\cdot P(M_{1}) = 0.38\cdot 0.3 = 0.114$ los sucesos no son independientes. 
    \item \underline{$N_{2}$ con $N_{1}$}: Como $P(N_{2}\cap N_{1}) = 0.08$ y $P(N_{2})\cdot P(N_{1}) = 0.38\cdot 0.1 = 0.038$ los sucesos no son independientes. 
\end{itemize}

\textbf{Ejercicio 3}


\end{document} 